% Created 2020-11-16 Mon 19:35
% Intended LaTeX compiler: pdflatex
\documentclass[11pt]{article}
\usepackage[utf8]{inputenc}
\usepackage[T1]{fontenc}
\usepackage{graphicx}
\usepackage{grffile}
\usepackage{longtable}
\usepackage{wrapfig}
\usepackage{rotating}
\usepackage[normalem]{ulem}
\usepackage{amsmath}
\usepackage{textcomp}
\usepackage{amssymb}
\usepackage{capt-of}
\usepackage{hyperref}
\date{\today}
\title{TD SIMD}
\hypersetup{
 pdfauthor={},
 pdftitle={TD SIMD},
 pdfkeywords={},
 pdfsubject={},
 pdfcreator={Emacs 27.1 (Org mode 9.3)}, 
 pdflang={English}}
\begin{document}

\maketitle
\tableofcontents

\uline{Directives}:

\begin{itemize}
\item Ce TD sera a rendre pour au plus tard Dimanche 00:00 (minuit) sur mon courriel UVSQ: \emph{mohammed-salah.ibnamar@uvsq.fr} avec comme objet: \textbf{\texttt{[AP TD\_simd - NOM PRENOM]}}
\end{itemize}

\uline{Notes}:

\begin{itemize}
\item Vous pouvez effectuer votre exploration sur une VM car l'execution du code n'est pas importante.
\item Evitez de copier du code, je le verrai tres rapidement :)
\end{itemize}

\section{Introduction}
\label{sec:orgad478f1}

L'objectif de ce TD est d'explorer le jeu d'instructions SIMD present 
sur les processeurs x86 (Intel et AMD) dont vous disposez.

\section{Travail a effectuer}
\label{sec:org1a77e54}
\subsection{Code}
\label{sec:org763d1f1}

Vous devez implementer plusieurs versions du code \textbf{dotprod} ci-dessous en deroulant autant de fois 
que vous desirez la boucle interne et comparer les codes assembleurs produits par le compilateur 
pour les differentes versions avec differentes options de compilation.

\begin{verbatim}

double dotprod(double *restrict a, double *restrict b, unsigned long long n)
{
    double d = 0.0;

    for (unsigned long long i = 0; i < n; i++)
       d += a[i] * b[i];

    return d;
}

\end{verbatim}

\subsection{Options de compilation}
\label{sec:orgae66d41}

\uline{Vous pouvez utiliser un ou plusieurs compilateurs, par exemple}: 

\begin{itemize}
\item GNU C Compiler \textbf{gcc}
\item Intel C Compiler \textbf{icc}
\item LLVM \textbf{clang}
\item AMD Optimizing C Compiler \textbf{aocc}.
\end{itemize}

\uline{Il faudra compiler le programme en suivant les etapes suivantes (pour \textbf{gcc} et \textbf{clang})}:

\begin{itemize}
\item \uline{Version de base}: -O1
\item \uline{Version legerement optimisee}: -O2
\item \uline{Version optimisee}: -O3
\item \uline{Version fortement optimisee}: -Ofast
\item \uline{Version kamikaze}: -march=native -mtune=native -Ofast -funroll-loops -finline-functions -flto -ftree-vectorize
\end{itemize}

\section{Rendu}
\label{sec:org60c2437}

Vous devez fournir une archive contenant les codes sources et les codes assembleurs analyses en plus d'un \textbf{README} qui fournit
des explications pour chacune des versions de la fonction et si les instructions SIMD x86 (jeux d'instructions SSE et AVX) ont ete 
utilisees de facon scalaire (i.e. add[sd] ou add[ss]) ou vectorielle (i.e. add[pd] ou add[ps]).

\begin{itemize}
\item\relax [ss | sd]: \textbf{ss} pour \textbf{scalar} \textbf{simple} precision et \textbf{sd} pour \textbf{scalar} \textbf{double} precision
\item\relax [ps | pd]: \textbf{ps} pour \textbf{packed} \textbf{simple} precision et \textbf{pd} pour \textbf{packed} \textbf{double} precision
\end{itemize}

Il est preferable de creer un \textbf{makefile} avec une regle par version.

Il faudra aussi fournir un fichier contenant la version du compilateur utilise:

\begin{verbatim}

#Pour GCC
$ gcc --version > gcc_ver.txt

#Pour clang
$ clang --version > clang_ver.txt

\end{verbatim}

Et un fichier contenant les informations sur votre processeur:

\begin{verbatim}

$ cat /proc/cpuinfo > cpuinfo.txt

\end{verbatim}

\section{References:}
\label{sec:org42d1b5f}

\begin{itemize}
\item \uline{RTFM}: man gcc

\item \url{https://gcc.godbolt.org/}
\item \url{https://software.intel.com/sites/landingpage/IntrinsicsGuide/}
\item \url{https://gcc.gnu.org/onlinedocs/gcc/Optimize-Options.html}
\end{itemize}
\end{document}
